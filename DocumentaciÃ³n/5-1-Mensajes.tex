%!TEX root = ejemplo.tex
%=========================================================
\section{Catálogo de mensajes}	
\label{sec:mensajes}

	En esta sección se describen todos los mensajes que aparecen en el sistema. Para cada mensaje se especifica:
	 
	\begin{description}\itemsep0em
		\item[Id:] Identificador del mensaje de la forma ``MSG XX'' y descripción corta del mismo.
		\item[Tipo:] Tipo del mensaje el cual puede ser: 
		\begin{description}
			\item[Normal:] Mensaje que informa al usuario una instrucción o el estado interno que guarda el sistema, suele tener un color {\color{msgNormalColor}Azul}.
			\item[Éxito:] Mensaje que informa al usuario sobre una acción realizada, sirve para confirmar el correcto funcionamiento del sistema. Se presentan con un color {\color{msgInfoColor}Verde}.
			\item[Atención:] Mensaje que tiene como finalidad llamar la atención del usuario a una situación que requiere su intervención, por ejemplo cuando una actividad ha generado un efecto colateral o se realizará una acción destructiva y no reversible. Se presentan con un color {\color{msgWarningColor}Naranja}.
			\item[Error:] Mensaje que informa al usuario un fallo en en una operación o un impedimento para realizarla, por ejemplo: cuando no se puede efectuar la acción solicitada, cuando un dato falta o tiene un formato no aceptado por el sistema. Se presentan con un color {\color{msgErrorColor}Rojo}.
		\end{description}
		\item[Propósito:] Explicación del propósito del mensaje.
		\item[Redacción:] Redacción del mensaje.
		\item[Parámetros:] En caso de que el mensaje pueda variar se especifican los casos y la forma en que debe adaptarse la redacción
		\item[Ejemplos:] Ejemplos de como debe renderizarse el mensaje.
	\end{description}


\subsection{Lista de mensajes}

%msgNormalColor
%msgInfoColor
%msgWarningColor
%msgErrorColor
\begin{cdtMessage}[msgInfoColor]{MSG-001}{Bienvenida al usuario}
	\item[Propósito:] Indicar al usuario que ha ingresado satisfactoriamente al sistema.
	\item[Redacción:] Bienvenido $<$nombre$>$.
	\item[Parámetros:] \hspace{1cm}
	\begin{itemize}
		\item $<$nombre$>$ \hyperlink{Usuario.nombre}{Nombre completo} del Usuario.
	\end{itemize}
	\item[Ejemplos:] Bienvenido Juan Pérez.
\end{cdtMessage}

%msgNormalColor
%msgInfoColor
%msgWarninigColor
%msgErrorColor
\begin{cdtMessage}{MSG-002}{Usuario no registrado} 
	\item[Propósito:] Indica que el usuario ingresado no existe en el sistema.
	\item[Redacción:] El usuario $<$login$>$ no se encuentra registrado.
	\item[Parámetros:] \hspace{1cm}
	\begin{itemize}
		\item $<$login$>$ \hyperlink{Usuario.login}{Login} del Usuario.
	\end{itemize}
	\item[Ejemplos:] El usuario juanP no se encuentra registrado.
\end{cdtMessage}

%msgNormalColor
%msgInfoColor
%msgWarninigColor
%msgErrorColor
\begin{cdtMessage}[msgErrorColor]{MSG-003}{Cuenta inactiva} 
	\item[Propósito:] Indicar al usuario que la cuenta especificada se encuentra inactiva.
	\item[Redacción:] La cuenta especificada $<$login$>$ se encuentra inactiva, favor de contactar al Secretario Escolar para mas información.
	\item[Parámetros:] \hspace{1cm}
	\begin{itemize}
		\item $<$login$>$ \hyperlink{Usuario.login}{Login} del Usuario.
	\end{itemize}
	\item[Ejemplos:] La cuenta especificada juanP se encuentra inactiva, favor de contactar al Secretario Escolar para mas información.
\end{cdtMessage}

%msgNormalColor
%msgInfoColor
%msgWarninigColor
%msgErrorColor
\begin{cdtMessage}[msgErrorColor]{MSG-004}{Error de inicio de sesión}
	\item[Propósito:] Indica al usuario que la contraseña introducida es incorrecta.
	\item[Redacción:] La contraseña ingresada es incorrecta.
	\item[Parámetros:] No aplica.
	\item[Ejemplos:] La contraseña ingresada es incorrecta.
\end{cdtMessage}

%msgNormalColor
%msgInfoColor
%msgWarninigColor
%msgErrorColor
\begin{cdtMessage}{MSG-008}{Tiempo restante para terminar un proceso} 
	\item[Propósito:] Indicar al usuario el tiempo restante para terminar una operación limitada en el tiempo como la reinscripción.
	\item[Redacción:] Quedan $<$tiempo$>$ para terminar la $<$operación$>$.
	\item[Parámetros:] \hspace{1cm}
	\begin{itemize}
		\item $<$tiempo$>$ Tiempo faltante para la operació especificando días, horas minutos y segundos.
		\item 
	\end{itemize}
	\item[Ejemplos:] \hspace{1cm}
	\begin{itemize}
		\item Quedan 45 días, 2 horas 12 minutos y 45 segundos para iniciar tu reinscripción.
		\item Quedan 2 minutos y 32 segundos para terminar tu reinscripción
	\end{itemize}
\end{cdtMessage}

\begin{cdtMessage}[msgInfoColor]{MSG-009}{Operación exitosa}
	\item[Propósito:] Informar al usuario que la operación solicitada ha sido ejecutada con éxito.
	\item[Redacción:] $<$Artículo$>$ $<$Operación$>$ del $<$Entidad$>$ $<$Identificador$>$ se realizó con éxito.
	\item[Parámetros:]\hspace{1pt}
	\begin{itemize}
		\item $<$Artículo$>$ $<$Operación$>$ se refiere a la operación realizada.
		\item $<$Entidad$>$ $<$Identificador$>$ se refiere al elemento del negocio donde recayó la operación, indicando el tipo del objeto y un dato que el usuario pueda usar para identificarlo.
	\end{itemize}
	\item[Ejemplo:] Algunos ejemplos son
	\begin{itemize}
		\item El registro del Alumno 342343 se realizó con éxito.
		\item La eliminación de la Tarea ``documentar el proceso'' se ha realizado con éxito.
	\end{itemize}
\end{cdtMessage}